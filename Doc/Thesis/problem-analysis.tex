\chapter{Problem analysis}
\label{goal}

The goal of the LinqToWiki library is to be able to express requests to the MediaWiki API
in a way that is readable, discoverable and checked by the compiler for correctness as much as possible.

This is achieved by generating classes specific for each module and using them in LINQ queries.

The two most commonly used variants of LINQ are LINQ to Objects, and various versions of LINQ for SQL databases.
LINQ to Objects is used for querying in-memory data, like arrays.
There are several widely-used libraries for accessing SQL databases using LINQ, including LINQ to SQL, LINQ to Entities and NHibernate.

In all versions of LINQ, the queries look the same. For example:

\begin{lstlisting}
from product in allProducts
where product.Price > 500
   && product.InStock
join category in categories on product.Category equals category
orderby product.Price
select product.Name
\end{lstlisting}

A query like this is translated into a sequence of method calls that take their parameters in the form of lambda expressions.
For example, the \lstinline{where} part of the above query is translated into:

\begin{lstlisting}
Where(product => product.Price > 500 && product.InStock)
\end{lstlisting}

The commonalities between LINQ to Objects and SQL LINQ libraries are that the full range of operators are available
and that all properties are available in all of them.

The situation with the MediaWiki API is different in several ways:

\begin{enumerate}
\item It doesn't support queries represented by many of the LINQ operators, including \lstinline{join} and \lstinline{group by}.
\item Some of the modules don't support sorting, some do. Of those that do support sorting, some allow specifying the sort key, others only direction.
\item The sets of properties that are available for filtering, sorting and selecting are all different.
\item There are modules used for queries about a set of pages. Those pages can be from a hard-coded list or a result from some other module.
\item There are also parameters that don't fit into the LINQ model well. Some of them are required, some are not.
\end{enumerate}

The goal is to be able to represent all valid queries, while invalid queries should cause a compile-time error.

Specifically, unsupported operators (like \lstinline{join} and \lstinline{group by}) should cause an error for all modules,
while the \lstinline{orderby} clause should cause an error only for the modules that don't support sorting.

Also, all operators should support only those properties that are actually supported by the API.
So, for example for the \verb,blocks, module, the following query should compile and execute fine:

\begin{lstlisting}
from block in wiki.Blocks()
where block.Ip == "8.8.8.8"
orderby block descending
select block.ById
\end{lstlisting}

While the following query should cause three errors:

\begin{lstlisting}
from block in wiki.Blocks()
where block.ById == 1234
orderby block.Expiry descending
select block.Ip
\end{lstlisting}

This is because limiting the query by the blocked IP address, sorting without specifying the key and selecting the ID of the user who performed the block are all allowed.
On the other hand, limiting by the ID of the user who performed the block, sorting by the expiration date and selecting the IP address are all impossible.
(Actually selecting the IP address of the blocked user is possible, but the information is contained in properties with different names.)