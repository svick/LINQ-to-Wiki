%%% Hlavní soubor. Zde se definují základní parametry a odkazuje se na ostatní části. %%%

%% Verze pro jednostranný tisk:
% Okraje: levý 40mm, pravý 25mm, horní a dolní 25mm
% (ale pozor, LaTeX si sám přidává 1in)
\documentclass[12pt,a4paper]{report}
\setlength\textwidth{145mm}
\setlength\textheight{247mm}
\setlength\oddsidemargin{15mm}
\setlength\evensidemargin{15mm}
\setlength\topmargin{0mm}
\setlength\headsep{0mm}
\setlength\headheight{0mm}
% \openright zařídí, aby následující text začínal na pravé straně knihy
\let\openright=\clearpage

\let\sectionnobreak = \section
\renewcommand{\section}[1]{
	\pagebreak
	\sectionnobreak{#1}
}

\usepackage[utf8]{inputenc}

%% Ostatní balíčky
\usepackage{graphicx}
\usepackage{amsthm}

\usepackage{upgreek}

\usepackage{listings}
\usepackage{tikz}
\usetikzlibrary{arrows}
\usepackage{xyling}
\usepackage{url}
\usepackage{microtype}
\usepackage{lmodern}
\usepackage{epstopdf}

\usepackage[center]{caption}

\usepackage{needspace}

\lstdefinelanguage{CSharp}[Sharp]{C}
{morekeywords={from,where,join,on,equals,orderby,descending,group,by,let,select},columns=flexible,showstringspaces=false}

\lstnewenvironment{code}[1][]%
{
   \noindent
   \minipage{\linewidth} 
   \vspace{0.5\baselineskip}
   \lstset{#1}}
{\endminipage}

\newcommand{\lstBreak}{\discretionary{\texttt{-}}{}{}}

\lstset{language=CSharp,basicstyle=\ttfamily}

%% Balíček hyperref, kterým jdou vyrábět klikací odkazy v PDF,
%% ale hlavně ho používáme k uložení metadat do PDF (včetně obsahu).
%% POZOR, nezapomeňte vyplnit jméno práce a autora.
\usepackage[unicode]{hyperref}   % Musí být za všemi ostatními balíčky
\hypersetup{pdftitle=.NET library for the MediaWiki API}
\hypersetup{pdfauthor=Petr Onderka}

\usepackage[all]{hypcap}

%%% Drobné úpravy stylu

% Tato makra přesvědčují mírně ošklivým trikem LaTeX, aby hlavičky kapitol
% sázel příčetněji a nevynechával nad nimi spoustu místa. Směle ignorujte.
\makeatletter
\def\@makechapterhead#1{
  {\parindent \z@ \raggedright \normalfont
   \Huge\bfseries \thechapter. #1
   \par\nobreak
   \vskip 20\p@
}}
\def\@makeschapterhead#1{
  {\parindent \z@ \raggedright \normalfont
   \Huge\bfseries #1
   \par\nobreak
   \vskip 20\p@
}}
\makeatother

% Toto makro definuje kapitolu, která není očíslovaná, ale je uvedena v obsahu.
\def\chapwithtoc#1{
\chapter*{#1}
\addcontentsline{toc}{chapter}{#1}
}

\begin{document}

\hypersetup{pageanchor=false}

% Trochu volnější nastavení dělení slov, než je default.
\lefthyphenmin=2
\righthyphenmin=2

%%% Titulní strana práce

\pagestyle{empty}
\begin{center}

\large

Charles University in Prague

\medskip

Faculty of Mathematics and Physics

\vfill

{\bf\Large BACHELOR THESIS}

\vfill

\centerline{\mbox{\includegraphics[width=60mm]{img/logo}}}

\vfill
\vspace{5mm}

{\LARGE Petr Onderka}

\vspace{15mm}

% Název práce přesně podle zadání
{\LARGE\bfseries .NET library for the MediaWiki API}

\vfill

% Název katedry nebo ústavu, kde byla práce oficiálně zadána
% (dle Organizační struktury MFF UK)
Department of Theoretical Computer Science \linebreak and Mathematical Logic

\vfill

\begin{tabular}{rl}

Supervisor of the bachelor thesis: & Tomáš Petříček \\
\noalign{\vspace{2mm}}
Study programme: & Computer Science \\
\noalign{\vspace{2mm}}
Specialization: & General Computer Science \\
\end{tabular}

\vfill

% Zde doplňte rok
Prague 2012

\end{center}

\newpage

%%% Následuje vevázaný list -- kopie podepsaného "Zadání bakalářské práce".
%%% Toto zadání NENÍ součástí elektronické verze práce, nescanovat.

%%% Na tomto místě mohou být napsána případná poděkování (vedoucímu práce,
%%% konzultantovi, tomu, kdo zapůjčil software, literaturu apod.)

\openright

\noindent
I would like to thank to my supervisor, Tomáš Petříček,
for his help with writing this thesis.
I would also like to thank to my family for their unending support and patience during my studies.

\newpage

%%% Strana s čestným prohlášením k bakalářské práci

\vglue 0pt plus 1fill

\noindent
I declare that I carried out this bachelor thesis independently, and only with the cited
sources, literature and other professional sources.

\medskip\noindent
I understand that my work relates to the rights and obligations under the Act No.
121/2000 Coll., the Copyright Act, as amended, in particular the fact that the Charles
University in Prague has the right to conclude a license agreement on the use of this
work as a school work pursuant to Section 60 paragraph 1 of the Copyright Act.

\vspace{10mm}

\hbox{\hbox to 0.5\hsize{%
In Prague, date ............
\hss}\hbox to 0.5\hsize{%
signature of the author
\hss}}

\vspace{20mm}
\newpage

%%% Povinná informační strana bakalářské práce

\vbox to 0.5\vsize{
\setlength\parindent{0mm}
\setlength\parskip{5mm}

Název práce:
.NET knihovna pro MediaWiki API
% přesně dle zadání

Autor:
Petr Onderka

Katedra:  % Případně Ústav:
Katedra teoretické informatiky a matematické logiky 
% dle Organizační struktury MFF UK

Vedoucí bakalářské práce:
Mgr. Tomáš Petříček, University of Cambridge
% dle Organizační struktury MFF UK, případně plný název pracoviště mimo MFF UK

Abstrakt:
% abstrakt v rozsahu 80-200 slov; nejedná se však o opis zadání bakalářské práce

Wiki běžící na systému MediaWiki poskytují svým uživatelům API,
které lze použít k přístupu k dané wiki z počítačového programu.
Toto API je rozsáhlé, často se mění a může se lišit wiki od wiki,
takže může být náročné napsat knihovnu pro přístup k tomuto API.

Tato práce popisuje LinqToWiki,
knihovnu pro přístup k MediaWiki API ze C\# nebo jiných jazyků na platformě .Net.
Díky použití LINQu a generovaní kódu pomocí Roslynu,
kód napsaný s použitím této knihovny je čitelný, objevitelný, silně typovaný a flexibilní.
Klíčová slova: Wiki, C\#, LINQ, Generování kódu, Roslyn
% 3 až 5 klíčových slov

\vss}\nobreak\vbox to 0.49\vsize{
\setlength\parindent{0mm}
\setlength\parskip{5mm}

Title:
.NET library for the MediaWiki API
% přesný překlad názvu práce v angličtině

Author:
Petr Onderka

Department:
Department of Theoretical Computer Science and Mathematical Logic
% dle Organizační struktury MFF UK v angličtině

Supervisor:
Mgr. Tomáš Petříček, University of Cambridge
% dle Organizační struktury MFF UK, případně plný název pracoviště
% mimo MFF UK v angličtině

Abstract:
% abstrakt v rozsahu 80-200 slov v angličtině; nejedná se však o překlad
% zadání bakalářské práce

MediaWiki wikis provide their users an API, that can used to programmatically access the wiki.
This API is large, changes frequently and can be different from wiki to wiki,
so it can be a challenge to write a library for accessing the API.

This thesis describes LinqToWiki, a library
that can be used to access the MediaWiki API from C\# or other .Net languages.
Thanks to the use of LINQ and code generation through Roslyn,
code written using this library is readable, discoverable, strongly-typed and flexible.

Keywords: Wiki, C\#, LINQ, Code generation, Roslyn
% 3 až 5 klíčových slov v angličtině

\vss}

\newpage

%%% Strana s automaticky generovaným obsahem bakalářské práce. U matematických
%%% prací je přípustné, aby seznam tabulek a zkratek, existují-li, byl umístěn
%%% na začátku práce, místo na jejím konci.

\hypersetup{pageanchor=true}
\openright
\pagestyle{plain}
\setcounter{page}{1}
\tableofcontents

%%% Jednotlivé kapitoly práce jsou pro přehlednost uloženy v samostatných souborech
\chapter*{Introduction}
\addcontentsline{toc}{chapter}{Introduction}

Wiki websites running on the MediaWiki software (such as Wikipedia) offer an API (Application programming interface)
for programmatic access to their database.
Since MediaWiki contains many functions, the API is quite extensive too: the core installation contains over seventy “modules”
and more are available through extensions. Each module accepts parameters in the form of key-value pairs
and returns a structured response in one of the possible formats, including XML.

Because of the size of the API, accessing it from some programming language is not very simple.
Two basic approchaches are possible: static and dynamic.

The dynamic approach is to create a thin library around the API modules:
let the user specify the names of the parameters and their types and return the response
in a dynamic manner, possibly as an associative array, or something like XML DOM.

This way, the user is responsible for the correctness of his query and for correct processing
of the response.
Also, it is hard to discover what are the possible parameters, what values can they have
and what form of response to expect.
This can lead to excessive use of the documentation or a ``trial and error'' approach.

The static approach is to create an extensive library that contains methods tailored for every module,
each returning a different statically-typed result.

This way, many of the errors the user could make will result in a compile-time error
and the development environment can also advise the user what options are available.

But this approach is also inflexible: if the user wants to use something the library
was not made for, he can't.
Differences like this can be caused by different versions of the software,
different sets of installed extensions, or just by different configuration.

Another question with the static approach is how to represent the parameters in code.
Most modules have many optional parameters, and so presenting them to the user
in an understandable manner might be a challenge.

One more problematic part might be how to represent choosing which properties to return in the result.
A list of strings representing the chosen properties might be suitable for the dynamic approach, but not so much for the static one.

\medskip

This work introduces the LinqToWiki library that tries to solve all those problems
in the form of a .Net library.

The dynamic vs. static issues are solved by automatically generating statically typed
code based on the metadata the API provides about itself.
The code generation is performed using Roslyn,
which is a new implementation of a compiler for the C\# language written in C\#.

The problems specific to the static approach are solved by using LINQ (Language integrated query):
a set of features of the C\# language and the .Net framework,
that is useful for representing queries and their translation into another form.

\chapter{Problem analysis}
\label{goal}

The goal of the LinqToWiki library is to be able to express requests using the MediaWiki \ac{API}
in a way that is readable, discoverable, checked by the compiler for correctness as much as possible and also flexible with regards to changes.

This is achieved by generating classes specific for each module and using them in LINQ queries.

\paragraph{Querying data in C\#}

\ac{LINQ} is a way of querying various data sources from the C\# language.
The two most commonly used variants are LINQ to Objects, and various versions of LINQ for \ac{SQL} databases.
LINQ to Objects is used for querying in-memory data, like arrays.
There are several widely-used libraries for accessing \ac{SQL} databases using LINQ, including LINQ to SQL, LINQ to Entities and NHibernate.

In all versions of \ac{LINQ}, the queries look the same. For example:

\begin{lstlisting}
from product in products
where product.Price > 500
   && product.InStock
join category in categories on product.Category equals category
orderby product.Price
select product.Name
\end{lstlisting}

A query like this is translated into a sequence of method calls that take their parameters in the form of lambda expressions.
For example, the \lstinline{where} part of the above query is translated into:

\begin{lstlisting}
products.Where(product => product.Price > 500 && product.InStock)
\end{lstlisting}

The commonalities between LINQ to Objects and SQL LINQ libraries are that the full range of operators is available
and that all properties of the queried type are available in all of them.

\paragraph{Querying MediaWiki}

The situation with the MediaWiki \ac{API} is different in several ways:

\begin{enumerate}
\item It does not support queries represented by many of the LINQ operators, including \lstinline{join} and \lstinline{group by}.
\item Some of the modules do not support sorting, some do. Of those that do support sorting, some allow specifying the sort key, others only the direction.
\item The sets of properties that are available for filtering, sorting and selecting are all different.
\item There are modules used for queries about a set of pages. Those pages can be from a hard-coded list or a result from some other module.
\item There are also parameters that do not fit into the LINQ model well. Some of them are required, some are not.
\end{enumerate}

The goal is to be able to represent all valid queries, while invalid queries should cause a compile-time error.

Specifically, unsupported operators (like \lstinline{join} and \lstinline{group by}) should cause an error for all modules,
while the \lstinline{orderby} clause should cause an error only for the modules that do not support sorting.

Also, all operators should support only those properties that are actually supported by the \ac{API}.
So, for example for the \verb,blocks, module, the following query should compile and execute fine:

\begin{lstlisting}
from block in wiki.Blocks()
where block.Ip == "8.8.8.8"
orderby block descending
select block.ById
\end{lstlisting}

This is because
\begin{compactitem}
\item limiting the query by the blocked \ac{IP} address,
\item sorting without specifying the key and
\item selecting the ID of the user who performed the block
\end{compactitem}
are all allowed, while the following query should cause three errors:

\begin{lstlisting}
from block in wiki.Blocks()
where block.ById == 1234
orderby block.Expiry descending
select block.Ip
\end{lstlisting}

Here,
\begin{compactitem}
\item limiting by the ID of the user who performed the block,
\item sorting by the expiration date and
\item selecting the \ac{IP} address
\end{compactitem}
are all impossible.
(Actually selecting the \ac{IP} address of the blocked user is possible,
but the information is contained in properties with different names.)

\section{Sample queries}

Following are samples of queries of different modules using LinqToWiki:

\begin{lstlisting}
string diff = wiki.compare(fromrev: 486474789,
                              torev: 487063697)
                    .value;
\end{lstlisting}

This query compares the text of two revisions and returns differences between them.
It shows that queries for modules that return a single item take their parameters as named method parameters
and directly return their result.

\begin{lstlisting}
var pages = wiki.Query.querypage(querypagepage.Uncategorizedpages)
                  .ToList();
\end{lstlisting}

This query returns a list of uncategorized pages,
as provided by the special page \texttt{Special:UncategorizedPages}.
It shows that required parameters to list-returning modules are given as method parameters
and that \lstinline{ToList()} (or, alternatively, \lstinline{ToEnumerable()})
has to be called on the result before it can be used.

\begin{lstlisting}
var pages = (from cm in wiki.Query.categorymembers()
	       where cm.title == "Category:Query languages"
	           && cm.type == categorymemberstype.subcat
	       select cm).Pages;
\end{lstlisting}

This query returns subcategories of the category \texttt{Query languages}.
The result is not actually directly usable, but it can be used in so called \texttt{prop} queries.
The query shows how LINQ can be used for querying list-returning modules
and that the \lstinline{Pages} property is used to create source for further queries.

\begin{lstlisting}
var result = pages.Select(
	p =>
	new
	{
		Info = p.info,
		Images = p.images().ToEnumerable()
	})
	.ToEnumerable();
\end{lstlisting}

This query returns information about images present on a list of pages specified by \lstinline{pages}
(see previous query).
It shows how the result of one query can be used as a source for another query,
how the LINQ method \lstinline{Select()} is used in such queries,
how anonymous types can be used to return the required information
and how \lstinline{ToEnumerable()} (or \lstinline{ToList()}) has to be also used inside this kind of queries.

\chapter{Background}
\section{MediaWiki API}

MediaWiki is an open source wiki system.
It is written in the PHP programming language and uses a relational database to store its data, usually MySQL.
It is maintained by the Wikimedia Foundation, who also runs some of the biggest wikis, including Wikipedia and Wiktionary.
It is also used by many others, including Wikia, who runs many small wikis for various interests and
the unofficial wiki of the Faculty of Mathematics and Physics, $\upomega\upiota\upkappa\upiota$.matfyz.cz.

\medskip

There are several ways to programmatically access the database of some MediaWiki wiki.
First, it's possible to directly access the database using SQL.
This usually requires access to the server that runs the database, so it's not available in many cases.
For Wikimedia wikis, a read-only access to most data, but excluding article texts,
is available for registered users of Toolserver, run by Wikimedia Deutschland.

Mostly specific to Wikimedia wikis is also another option: database dumps. These are files that contain
dumps of some tables of the wikis. Their disadvantages are that the newest dump is usually several days or weeks old
and that the files can be huge, which is impractical for getting information about a small number of pages.

Last, but not least, is the MediaWiki API.
It can be used to remotely access any MediaWiki wiki (unless the API is disabled in the configuration)
using the HTTP protocol.

\medskip

Parameters for an API request are given in the query string of a GET request or in the body of a POST request
(modules that perform modifications require the use of POST).
The body of the POST request is usually formatted as \path{application/x-www-form-urlencoded},
but file uploads require the use of \path{multipart/form-data}.

Some parameters can accept multiple values at once.
In these cases, the values are separated by a pipe character (\texttt{|}).

There are some parameters that are common to many modules:

\begin{itemize}
\item The \texttt{prop} parameter is used to determine what properties will be present in the response.
The values of this parameter don't map directly to the properties of the response,
so for example specifying \texttt{prop=ids} might cause the property \texttt{pageid} to appear in the result.
\item The \texttt{dir} parameter is used to determine the order of the results:
whether it should be ascending or descending.
\item The \texttt{sort} parameter decides which property will be used to order the results of the query.
\end{itemize}

The response can be in one of the several available formats, the most widely used ones are XML and JSON.

The representation of most data types in the response is nothing unusual:
\texttt{string}s are formatted as strings, \texttt{integer}s as decimal numbers,
\texttt{timestamp}s are formatted according to ISO 8601.
Only \texttt{boolean}s have a possibly unexpected representation:
if the property is \texttt{false}, it is not present in the result at all,
and if its value is \texttt{true}, it is represented as an empty string.

If there is some problem executing a request, for example if a parameter has an invalid value,
a warning will be returned along with the result of the operation.
In the case of a fatal problem, such as when the user doesn't have the right to perform an action,
an error is returned, without any results.

\medskip

The API is divided into modules and there are two kinds of modules:
“normal” modules (called “non-query modules” in this work) and query modules.

Non-query modules are usually used to perform some action.
For example the \texttt{edit} module can be used to edit a page
and the \texttt{block} module can be used to block another user (it can be used only by users with sufficient privileges).

Query modules are used for retrieving information about the wiki. There are three types of query modules:

\begin{itemize}
\item \texttt{list} modules: Return contents of various lists.
For example the \texttt{all\-categories} module can be used to list all categories on a wiki,
while the \texttt{categorymembers} module can be used to list members of a certain category.
\item \texttt{prop} modules: Return information about a set of pages.
For example, the \texttt{categories} module can be used to retrieve the categories for each page in a given set.
\item \texttt{meta} modules: Return meta information that are not directly associated with pages.
For example the \texttt{userinfo} module can be used to retrieve information about the currently logged-in user.
\end{itemize}

For \texttt{prop} modules, the set of pages they operate on can be specified directly using page titles or page IDs.

Another option is to use some other module (usually a \texttt{list} module) as a so called “generator”.
This way, one can for example retrieve all categories of pages in a specific category,
by using the \texttt{categorymembers} module as a generator for the \texttt{categories} module.

Because more than one module can be used in one request,
the parameters for each module are distinguished by using prefixes.
For example, the prefix for the \texttt{categorymembers} module is \texttt{cm}.
So, setting its \texttt{limit} parameter to the value of~5 can be achieved by
adding \texttt{cmlimit=5} to the query string of a GET request or to the body of a POST request.

The API is also extensible: MediaWiki extensions can add their own modules and modify some behavior of existing modules.

An example of an API request URI and a response in the XML format is in Figure \ref{API example}.

\begin{figure}[htbp]
\texttt{http://en.wikipedia.org/w/api.php}~\texttt{?}\
\texttt{format}~\texttt{=}~\texttt{xml}~\texttt{\&}
\texttt{action}~\texttt{=}~\texttt{query}~\texttt{\&}
\texttt{list}~\texttt{=}~\texttt{categorymembers}~\texttt{\&}
\texttt{cmtitle}~\texttt{=}~\texttt{Category:Query\%20languages}~\texttt{\&}
\texttt{cmprop}~\texttt{=}~\texttt{title}~\texttt{\&}
\texttt{cmtype}~\texttt{=}~\texttt{page}~\texttt{\&}
\texttt{cmdir}~\texttt{=}~\texttt{descending}~\texttt{\&}
\texttt{cmlimit}~\texttt{=}~\texttt{5}

\begin{lstlisting}[language=xml]
<?xml version="1.0"?>
<api>
  <query>
    <categorymembers>
      <cm ns="0" title="YQL (programming language)" />
      <cm ns="0" title="Yahoo! query language" />
      <cm ns="0" title="XQuery" />
      <cm ns="0" title="XPath" />
      <cm ns="0" title="XBase++" />
    </categorymembers>
  </query>
  <query-continue>
    <categorymembers cmcontinue="page|5842415345|572327" />
  </query-continue>
</api>
\end{lstlisting}

\caption{An example of an API request and a response}
\label{API example}
\end{figure}

\subsection{Paging}
\label{mw paging}

Because the results of the API queries can contain thousands and sometimes even millions of entries,
the responses are limited.
For most modules, the default limit (when it is not specified as a parameter) is ten entries per page
and the default maximum is 500~entries for normal users.
For users with the \texttt{apihighlimits} right, the limits are raised, usually to 5000~entries per page.

In the \texttt{limit} parameter, one can specify either the exact value,
or the special value \texttt{max}, which means the maximum allowed for the current user.

To get the data from the following page, one has to use a value specified in the \texttt{query-continue}
element in the result (see Figure \ref{API example} again).
The value in this element is a transparent identifier of the next page.

The advantage of this system when compared with the conventional paging systems of numbering pages
or using numeric offsets is that it avoids missing entries and duplicates when the result
changes while retrieving the pages.

The API has no notion of transactions, so it is not possible to get fully consistent results
that would correspond to an exact moment in time.
But thanks to this paging system, one can be certain that an entry that should be in the result set
during retrieving of all of the pages will actually be present in the result set exactly once.

\medskip

The situation gets more complicated when using a \texttt{prop} module with another module as a generator.
That is because both modules have their own paging.

When such a request is made, the first response will contain a limited number of items from the generator
and a limited number of results from the \texttt{prop} module for those items.
To retrieve the next set of items from the generator, one has to use the \texttt{query-continue} for the generator
(called “primary paging” in this work).
To retrieve the next set of results for the items from the first result,
one has to use the \texttt{query-continue} for the \texttt{prop} module (called “secondary paging” here).

For an example, see Figure \ref{paging}.
It shows how the paging might work when using the \texttt{allpages} module as a generator,
together with the \texttt{prop} module \texttt{categories}.
The \texttt{query-continue} elements are not shown in the figure.

\medskip

\begin{figure}[tbp]
\begin{center}
\begin{tikzpicture}
\path (0,12) node(11) {
\begin{minipage}{150pt}
\begin{lstlisting}[language=xml,frame=single]
<query>
  <pages>
    <page title="A" />
    <page title="B">
      <categories>
        <cl title="X" />
        <cl title="Y" />
      </categories>
    </page>
  </pages>
</query>
\end{lstlisting}
\end{minipage}
} (7,12) node(12) {
\begin{minipage}{150pt}
\begin{lstlisting}[language=xml,frame=single]
<query>
  <pages>
    <page title="A">
      <categories>
        <cl title="X" />
        <cl title="Y" />
      </categories>
    </page>
    <page title="B" />
  </pages>
</query>
\end{lstlisting}
\end{minipage}
} (11,12) node(13) {\dots};

\path (0,4) node(21) {
\begin{minipage}{150pt}
\begin{lstlisting}[language=xml,frame=single]
<query>
  <pages>
    <page title="C">
      <categories>
        <cl title="X" />
        <cl title="Y" />
      </categories>
    </page>
    <page title="D" />
  </pages>
</query>
\end{lstlisting}
\end{minipage}
} (7,4) node(22) {
\begin{minipage}{150pt}
\begin{lstlisting}[language=xml,frame=single]
<query>
  <pages>
    <page title="C">
      <categories>
        <cl title="Z" />
      </categories>
    </page>
    <page title="D">
      <categories>
        <cl title="X" />
      </categories>
    </page>
  </pages>
</query>
\end{lstlisting}
\end{minipage}
};

\path (0,0) node(31) {\dots};

\draw[->] (11) -- (12);
\draw[->] (12) -- (13);
\draw[->] (11) -- (21);
\draw[->] (21) -- (22);
\draw[->] (21) -- (31);

%TODO: divné mezery
\end{tikzpicture}
\end{center}

\caption{An example of primary and secondary paging}
\label{paging}

\end{figure}

The situation is even more complicated with the \texttt{prop} module \texttt{revisions}.
It can be used to retrieve information about revisions of pages, including their text
and it is the only module that can be used to get the text of a set of pages.

For other modules, when no \texttt{limit} parameter is specified, a default value is used (usually 10)
and a \texttt{query-continue} element is present in the response, to access the remaining items.

But for the \texttt{revisions} module, not specifying the \texttt{limit} parameter means that only the most
recent revision will be shown and no \texttt{query-continue} will be present.
Also, when \texttt{limit} is specified, the module can operate only on one page at a time,
so for example one has to set the \texttt{limit} of a module used as a generator to~1.

\subsection{The \texttt{paraminfo} module}
\label{paraminfo}

A special importance for this work has the \texttt{meta} query module \texttt{paraminfo}.
This module can be used to retrieve information about modules,
which is necessary for generating code to access those modules in a static fashion.

Before this work, the \texttt{paraminfo} module provided some general information about the module
and, most importantly, information about parameters, their data types and a short description,
useful as a documentation for the generated code.

The data type of a parameter is either a simple type (e.g. \texttt{integer} or \texttt{string}),
or an enumeration of possible values.

A shortened example of a response from the \texttt{paraminfo} module
for the \texttt{categorymembers} module is in Figure~\ref{paraminfo sample}.

For code generation in LinqToWiki, another piece of information is necessary:
knowing the properties of the response and how do they map to the values of the \texttt{prop} parameter.
For information about how we added them, see Chaper~\ref{mw improvements}.

\begin{figure}[p]

\begin{lstlisting}[language=]
<module name="categorymembers" prefix="cm" querytype="list"
  generator="" listresult="" description="List all pages in a ...">
  <parameters>
    <param name="title" type="string" 
      description="Which category to enumerate (required). ..." />
    <param name="pageid" type="integer" 
      description="Page ID of the category to enumerate. ..." />
    <param name="prop" default="ids|title" multi=""
      description="What pieces of information to include ...">
      <type>
        <t>ids</t>
        <t>title</t>
        <t>sortkey</t>
        <t>sortkeyprefix</t>
        <t>type</t>
        <t>timestamp</t>
      </type>
    </param>
    <param name="namespace" multi="" type="namespace"
      description="Only include pages in these namespaces" />
    <param name="continue" type="string"
      description="For large categories, give the value ..." />
    <param name="limit" default="10" max="500" type="limit" 
      description="The maximum number of pages to return." />
    <param name="sort" default="sortkey"
      description="Property to sort by">
      <type>
        <t>sortkey</t>
        <t>timestamp</t>
      </type>
    </param>
    <param name="dir" default="ascending"
      description="In which direction to sort">
      <type>
        <t>ascending</t>
        <t>descending</t>
      </type>
    </param>
  </parameters>
</module>
\end{lstlisting}

\caption{A shortened response of the \texttt{paraminfo} module \\ for the \texttt{categorymembers} module}
\label{paraminfo sample}
\end{figure}
\section{LINQ and expression trees}

LINQ, short for Language Integrated Query, is a feature of the C\# programming language%
\footnote{Visual Basic .Net also supports LINQ, with slightly different syntax
and capabilities, but uses the same types.}
and the .Net framework that can be used for querying of various data sources.
It uses higher-order functions and lambda expressions to achieve a readable declarative syntax.

LINQ consists of a set of so called “standard query operators”:
methods that are used to perform the query operations on a given source.
Also, a special syntax (called “query expressions”), similar to SQL queries, is available
for some of those operators.
The compiler translates a query expression into a set of calls to standard query operators,
using lambda expressions and anonymous types.

Anonymous types are types that don't have to be explicitly declared;
they are used in similar situations as tuples in functional programming.
An instance of an anonymous type is created by using the \lstinline{new} keyword
without specifying the type of the object to create.

For example the following query expression (as seen in Chapter \ref{goal}):

\nopagebreak

\begin{code}
from product in allProducts
where product.Price > 500
   && product.InStock
join category in categories on product.Category equals category
orderby product.Price
select product.Name
\end{code}

Is translated into the following method calls:

\begin{code}
allProducts
    .Where(product => product.Price > 500 && product.InStock)
    .Join(
        categories,
        product => product.Category,
        category => category,
        (product, category) => new { product, category })
    .OrderBy(t => t.product.Price)
    .Select(t => t.product.Name)
\end{code}

The parameter~\lstinline,t, is called “transparent identifier”. It is used to transfer a set of variables from one method call to another.

The LINQ library also contains methods that do not have a corresponding representation in query expressions. Some examples of those are \lstinline,Aggregate(),, \lstinline,Sum(), and \lstinline,ToList(),.

Many of the basic query operators also correspond to a well-known higher-order functions from functional programming. See Figure \ref{LINQ methods} for comparison of some of the LINQ query operators, query expression clauses, and higher-order functions.

\begin{figure}[htbp]
\begin{tabular}{lll}
query operator & query expression clause & higher-order function \\
\lstinline,Select(), & \lstinline,select,, \lstinline,let, & map \\
\lstinline,Where(), & \lstinline,where, & filter \\
\lstinline,SelectMany(), & second and following \lstinline,from, & bind \\
\lstinline,Aggregate(), & & fold \\
\lstinline,Join(), & \lstinline,join, & \\
\lstinline,OrderBy(),,\cr \lstinline,OrderByDescending(), & \lstinline,orderby, & \\
\lstinline,GroupBy(), & \lstinline,group by, & \\
\lstinline,Sum(), \\
\lstinline,First(), \\
\lstinline,ToList(), \\
\end{tabular}

\caption{Comparison between LINQ query operators, query expression clauses, and higher-order functions}
\label{LINQ methods}
\end{figure}

Usually, lambda expressions are compiled into normal methods and passed to the query operator methods as delegates
(which are similar to function pointers in~C or first-class functions in functional languages).
But this would not be suitable for querying of sources that are not in-memory collections.
This is because the query has to be translated into another form,
like an SQL query or a set of parameters for the MediaWiki API.

Because of this, a lambda expression in C\# can be also compiled into another form:
an expression tree.
Expression tree is an object that represents the given lambda expression
in a form similar to an abstract syntax tree.
This object can be programmatically accessed and manipulated,
which allows translation of LINQ queries into other forms, such as SQL queries.
An expression tree can also be compiled into a delegate.

For an example of an expression tree, see Figure \ref{Expression tree}.

\begin{figure}[htbp]
\begin{center}
\Treek[2]{4}{
 & & & \K{Binary: AndAlso} \B{dll}_{\textnormal{Left}} \B{dr}^{\textnormal{Right}} \\
 & \K{Binary: GreaterThan} \B{dl}_{\textnormal{Left}} \B{dr}^{\textnormal{Right}} & & & \K{Member: InStock} \B{d}^{\textnormal{Expression}} \\
 \K{Member: Price} \B{d}_{\textnormal{Expression}} & & \K{Constant: 500} & & \K{Parameter: product} \\
\K{Parameter: product}
}
\end{center}

\caption{The body of the expression tree for the lambda expression \lstinline,product => product.Price > 500 && product.InStock,}
\label{Expression tree}
\end{figure}

The .Net framework contains two implementations of the query expression pattern:
the interfaces \lstinline,IEnumerable<T>, and \lstinline,IQueryable<T>,.
This means that any object that implements one of these two interfaces can be used in a LINQ query.

These two types implement the query expression pattern completely,
so they can be used with any LINQ operator.
Other custom types can implement only part of the query expression pattern,
which would mean only a subset of the LINQ operators are available for such types.

The \lstinline,IEnumerable<T>, interface usually represents an in-memory collection,
so its implementation of the LINQ operators use delegates.
The \lstinline,IQueryable<T>, interface is usually used to represent a remote collection
(such as a table in a relational database),
so its version of the LINQ operators use expression trees.

The \lstinline,IQueryable<T>, interface doesn't perform any translation of expression trees
into the target query language.
What it does is to combine the whole query into one expression tree,
which is then passed to an implementation of \lstinline,IQueryProvider,.

The query provider is then responsible for processing the expression tree
and translating it into its target query language.
If the query is not valid, the query provider will throw an exception at runtime.
\section{Roslyn}
\label{roslyn}

Microsoft Roslyn is a new implementation of the C\# compiler written in C\#
(and a \ac{VB.NET} compiler written in \ac{VB.NET}) \cite{roslyn}.
Its main distinguishing characteristic is that it is “open”:
it can be used for example to convert between text and a syntax tree,
to manipulate the syntax tree or to interrogate semantic information.

It also integrates itself into the Microsoft Visual Studio \ac{IDE},
where it can be used to perform
custom refactoring actions or to produce custom errors and warnings at compile-time.

Roslyn is currently under development and so far it had two public releases.
Both were in the form of \ac{CTP},
the first one from October 2011, the second one from June 2012.

In the second \ac{CTP}, the syntactic part of the library in completely implemented,
so for example the syntax tree can represent any construct of C\#
and any syntax tree can be translated to and from source code.
On the other hand, the semantic part of the library is not fully implemented,
which means that for example some syntax trees won't successfully compile,
even if they represent valid C\# code.

Because of its close relation with Visual Studio,
Roslyn syntax tree is able to represent every feature of C\# with down to character precision.
This includes “trivia”: parts of code that are not significant for the compiler,
such as whitespace and comments.

Trivia can also be “structured”, that is, it can form a small syntax tree of its own.
An example of structured trivia are \ac{XML} documentation comments,
that can be used to provide documentation for a piece of code,
which can then be automatically processed.

For an example of a Roslyn syntax tree, see Figure~\ref{Syntax tree}

\begin{figure}[htbp]

\begin{lstlisting}
public abstract CategoryInfoResult CategoryInfo { get; }
\end{lstlisting}

\begin{center}
\Treek[.5]{4}{
& & & & \K{Property}\Below{Declaration} \B[-5]{dllll} \B[-5]{dll} \B[-5]{d} \B[-5]{drr} \B[-5]{drrrr} \\
\K{Public}\Below{Keyword} & & \K{Abstract}\Below{Keyword} & & \K{Identifier}\Below{Name} \B[-5]{d} & & \K{Identifier}\Below{Token:}\Below{CategoryInfo} & & \K{Accessor}\Below{List} \B[-5]{dll} \B[-5]{d} \B[-5]{drr} \\
& & & & \K{Identifier}\Below{Token:}\Below{CategoryInfoResult} & & \K{OpenBrace}\Below{Token} & & \K{GetAccessor}\Below{Declaration} \B[-5]{dl} \B[-5]{dr} & & \K{CloseBrace}\Below{Token} \\
& & & & & & & \K{Get}\Below{Keyword} & & \K{Semicolon}\Below{Token}
}
\end{center}

\caption[Example of piece of C\# code and its Roslyn syntax tree]
{Example of piece of C\# code and its Roslyn syntax tree \\ (trivia not shown)}
\label{Syntax tree}
\end{figure}

\pagebreak[0]

Roslyn syntax trees are immutable and can be created using factory methods from the \lstinline{Syntax} class.
And while not all elements of the syntax tree have to be specified (like braces of a property accessor list),
creating a syntax tree can be quite cumbersome.

The exact syntax for creating syntax trees changed between the two \acp{CTP}.
In the October 2011 \ac{CTP}, methods with many optional parameters were used.
In the June 2012 \ac{CTP}, the situation somewhat improved:
the factory method now has parameters only for required children of the created node
and optional child nodes can be added in a fluent manner using \lstinline{With*} methods.

For an example of code to manually create the syntax tree from Figure~\ref{Syntax tree},
see Figure~\ref{Roslyn code 2011} for the October 2011 \ac{CTP} version and
Figure~\ref{Roslyn code 2012} for the June 2012 \ac{CTP} version.

\begin{figure}[p]

\begin{lstlisting}
Syntax.PropertyDeclaration(
  modifiers:
    Syntax.TokenList(
      Syntax.Token(SyntaxKind.PublicKeyword),
      Syntax.Token(SyntaxKind.AbstractKeyword)),
  type: Syntax.ParseTypeName("CategoryInfoResult"),
  identifier: Syntax.Identifier("CategoryInfo"),
  accessorList:
    Syntax.AccessorList(
      accessors:
        Syntax.List(
          Syntax.AccessorDeclaration(
            SyntaxKind.GetAccessorDeclaration,
            semicolonTokenOpt:
              Syntax.Token(SyntaxKind.SemicolonToken)))))
\end{lstlisting}

\caption{Sample code to manually create a Roslyn syntax tree \\ using October 2011 \ac{CTP}}
\label{Roslyn code 2011}
\end{figure}

\begin{figure}[p]

\begin{lstlisting}
Syntax.PropertyDeclaration(
  Syntax.ParseTypeName("CategoryInfoResult"),
  "CategoryInfo")
  .WithModifiers(
    Syntax.TokenList(
      Syntax.Token(SyntaxKind.PublicKeyword),
      Syntax.Token(SyntaxKind.AbstractKeyword)))
  .WithAccessorList(
    Syntax.AccessorList(
      Syntax.List(
        Syntax.AccessorDeclaration(
          SyntaxKind.GetAccessorDeclaration)
          .WithSemicolonToken(
            Syntax.Token(SyntaxKind.SemicolonToken)))))
\end{lstlisting}

\caption{Sample code to manually create a Roslyn syntax tree \\ using June 2012 \ac{CTP}}
\label{Roslyn code 2012}
\end{figure}

\chapter{MediaWiki improvements}
\label{mw improvements}

As mentioned in Section~\ref{paraminfo}, to generate types for each module of the \ac{API},
it is necessary to know the properties contained in the module response
and how do they map to the values of the \texttt{prop} parameter.

This information was not available, in MediaWiki previously.
For this reason, we extended the \texttt{paraminfo} module
to be able to provide information about result properties
of the \ac{API} modules, using the same type system already used to describe parameters.
Also, most of the \ac{API} modules were changed so that they provide this information to the \texttt{paraminfo} module.

Specifically, this meant adding \texttt{getResultProperties()} function to the \texttt{Api\-Base} class,
which is a base class for all module classes, and then overriding it in each module's class.
This function is then called from the \texttt{ApiParamInfo} class,
so that the provided information shows in the output of the \texttt{paraminfo} module.
For an example implementation of \texttt{getResultProperties()}, see Figure~\ref{paraminfo code}.

Of the 73~modules present in the MediaWiki core (that is, without any extensions),
5 are not suitable for having their result properties described,
because their result looks different than the result of other modules (for example, there are modules that produce \acs{RSS} feeds).
Further 5 modules do use the same response format as the other modules,
but their response cannot be described in the type system used.
There are also 17 modules that can be partially represented using this type system, but not completely.

The patch that adds this ability to the \texttt{paraminfo} module and the necessary
information to most other modules was reviewed by MediaWiki developers and merged into the official repository
on 12 June 2012.
On 2 July 2012, MediaWiki version 1.20wmf6, which includes changes from this patch, was deployed to all Wikimedia sites, including Wikipedias.

An example of the added result information to the \texttt{paraminfo} response (here for the \texttt{categorymembers} module) is in Figure~\ref{paraminfo props}.

\begin{figure}[htbp]
\begin{lstlisting}[language=php]
public function getResultProperties() {
  return array(
    'ids' => array(
      'pageid' => 'integer'
    ),
    'title' => array(
      'ns' => 'namespace',
      'title' => 'string'
    ),
    'sortkey' => array(
      'sortkey' => 'string'
    ),
    'sortkeyprefix' => array(
      'sortkeyprefix' => 'string'
    ),
    'type' => array(
      'type' => array(
        ApiBase::PROP_TYPE => array(
          'page',
          'subcat',
          'file'
        )
      )
    ),
    'timestamp' => array(
      'timestamp' => 'timestamp'
    )
  );
}
\end{lstlisting}

\caption{Implementation of \texttt{getResultProperties()} \\
in the \texttt{Api\-Query\-Category\-Members} class}
\label{paraminfo code}
\end{figure}

\begin{figure}[p]

\begin{lstlisting}[language=]
<props>
  <prop name="ids">
    <properties>
      <property name="pageid" type="integer" />
    </properties>
  </prop>
  <prop name="title">
    <properties>
      <property name="ns" type="namespace" />
      <property name="title" type="string" />
    </properties>
  </prop>
  <prop name="sortkey">
    <properties>
      <property name="sortkey" type="string" />
    </properties>
  </prop>
  <prop name="sortkeyprefix">
    <properties>
      <property name="sortkeyprefix" type="string" />
    </properties>
  </prop>
  <prop name="type">
    <properties>
      <property name="type">
      <type>
        <t>page</t>
        <t>subcat</t>
        <t>file</t>
      </type>
      </property>
    </properties>
  </prop>
  <prop name="timestamp">
    <properties>
      <property name="timestamp" type="timestamp" />
    </properties>
  </prop>
</props>
\end{lstlisting}

\caption{Result properties information for the \texttt{categorymembers} module}
\label{paraminfo props}
\end{figure}

During this work, we also noticed several bugs and inconsistencies in the \ac{API}.
Because of this, we reported eight bugs to the WikiMedia bug-tracking system.
Three of them turned out to be duplicates of already reported bugs and,
as of December 2012, three of them are still waiting to be fixed.

We also submitted eight additional patches to the MediaWiki code review system.
Although only three of them actually fix behavior of the MediaWiki \ac{API},
the rest are only fixes in documentation and other mostly insignificant changes.
Of those three patches, one is still waiting for review, because it is a breaking change,
and most likely will not be accepted.
\chapter{The LinqToWiki library}

The LinqToWiki library consists of one Visual Studio solution, that contains the following projects:

\begin{itemize}
\item LinqToWiki.Core
\item LinqToWiki.Codegen
\item LinqToWiki.Codegen.App
\item LinqToWiki.ManuallyGenerated
\item LinqToWiki.Samples
\end{itemize}

The LinqToWiki.Core project contains the core of the library:
types that access the API, convert to and from the representation of data in the API,
represent parameters of various types of queries, represent query results
or those that process LINQ expression trees.
This project can be used together with code generated using LinqToWiki.Codegen,
or with manually written code.

The LinqToWiki.Codegen project handles generating code based on information from the \texttt{paraminfo} module.
It contains types that represent the results of that module, process them, generate C\# code and compile this code.
This project also contains helper types for easier creating of Roslyn syntax trees.

The LinqToWiki.Codegen.App project compiles down to a simple console application called linqtowiki-codegen,
that uses functionality from the LinqToWiki.\allowbreak{}Codegen project.

The LinqToWiki.ManuallyGenerated project is a sample of how one could write code to access a wiki using LinqToWiki without using LinqToWiki.Codegen to generate the code.

Finally, the LinqToWiki.Samples project contains samples showing how to use various API modules using LinqToWiki.
It uses code generated by LinqToWiki.\allowbreak{}Codegen.\allowbreak{}App.

\medskip

The intended usage of LinqToWiki is this:
First run the linqtowiki-codegen application to generate a DLL library tailored for a certain wiki.
Then use the generated library together with LinqToWiki.Core in your C\# (or VB.NET) application to access that wiki.

Other options are possible, though.
For example, the LinqToWiki.Codegen library can be used to generate the code as a set of files containing C\# source code.
Those files can then be modified and manually compiled.

\section{The LinqToWiki.Core project}

The LinqToWiki.Core project contains shared code that can be used when querying any MediaWiki wiki
that has the API enabled.
It can be used together with code generated through LinqToWiki\allowbreak{}.Codegen,
but it can also be used without it.

In fact, LinqToWiki.\allowbreak{}Codegen internally uses LinqToWiki.Core to access the \texttt{paraminfo} module
using manually written code.

\subsection{\texorpdfstring{\lstinline{QueryTypeProperties}}{QueryTypeProperties}}

The \lstinline{QueryTypeProperties} class holds basic information about a “query type”,
which corresponds to an API module.
This information includes the prefix this module uses in its parameters,
what type of module it is or mapping of its result properties to values accepted by the \texttt{prop} parameter.
It is also able to parse XML elements this module returns.

\subsection{\texorpdfstring{\lstinline{WikiQuery}}{WikiQuery}}

Probably the most often used and certainly the most interesting queries are those using \texttt{list} query modules.
Such queries are represented in LinqToWiki by a group of types whose names start with \lstinline{WikiQuery}.

Specifically, there are four such types:
\lstinline{WikiQuery}, \lstinline{WikiQuerySortable}, \lstinline{Wiki}\lstBreak\lstinline{Query}\lstBreak\lstinline{Generator} and \lstinline{WikiQuerySortableGenerator}.
If a module supports sorting, it is represented by a type with \lstinline{Sortable} in its name
and if it supports being used as a generator for \texttt{prop} queries, it is represented by a type with \lstinline{Generator} in its name.

There is also a fifth type: \lstinline{WikiQueryResult}.
This type by itself represents a query that can't be modified anymore,
but can be used to execute it and get the results.
All of the four preceding types inherit from \lstinline{WikiQueryResult},
so it is possible  to execute the query using any one of them too.

The type governs what operations are available.
For example, if a type is one of the two \lstinline{Sortable} types,
it will have an \lstinline{OrderBy()} method, but no other type has this method.
Each method can also return a different type, as is necessary to form queries.

\medskip

All of the \lstinline{WikiQuery}-related types are generic
and their type parameters are used to decide what properties can be used in each operation.
For example, the type parameter \lstinline{TOrderBy} of \lstinline{WikiQuerySortable}
decides what properties can be used in the parameter of the \lstinline{OrderBy} method.

The way this is achieved is that \lstinline{TOrderBy} is a type that contains the properties that can be
used for sorting in the module \lstinline{WikiQuerySortable} represents
and the \lstinline{OrderBy} method accepts lambda expressions whose parameter is of this type.

For example, if some module supported sorting by \lstinline{PageId} and \lstinline{Title},
then \lstinline{TOrderBy} would be a type that contains two properties with those names.
Because of this, a query like \lstinline{source.OrderBy(x => x.Title)} would compile and execute fine,
but \lstinline{source.OrderBy(x => x.Name)} would fail to compile.

Because of the way lambda expressions work, queries like \lstinline[breaklines=true]{source.OrderBy(x => x.Title.Substring(1))} or \lstinline{source.OrderBy(x => random.Next())} would compile fine.
But because there is no way to efficiently execute such queries using the MediaWiki API,
they will fail with an exception at runtime.

\medskip

The various methods available on the \lstinline{WikiQuery} types are:

\begin{itemize}
\item \lstinline{Where()} only sets some parameter or parameters of a query,
it always returns the same type.

It is available on all four of the basic \lstinline{WikiQuery} types
and uses the generic type parameter \lstinline{TWhere}.

\item \lstinline{Select()} is used to choose how the elements in the resulting collection should look like
and what properties should they contain.
Because the result of the lambda passed into this method can be an arbitrary type,
it doesn't make sense to modify the query after calling this method.
Because of that, \lstinline{Select()} returns \lstinline{WikiQueryResult}.
This also follows query expression syntax, where \lstinline{select} is the last clause of each query.

It is available on all four of the \lstinline{WikiQuery} types
and uses the type parameter \lstinline{TSelect}.

\item \lstinline{ToEnumerable()} and \lstinline{ToList()} are used to actually execute the query.
The distinction between the two methods is that \lstinline{ToEnumerable()} returns an \lstinline{IEnumerable},
that lazily loads new pages of results on demand.
\lstinline{ToList()}, on the other hand, returns a \lstinline{List},
that is immediately loaded with all of the results, possibly from many pages.

These two methods are available on all of the \lstinline{WikiQuery} types, including \lstinline{WikiQueryResult}
and return the result based on the type parameter \lstinline{TSource} for most of the types.
And exception is \lstinline{WikiQueryResult}, which uses a separate \lstinline{TResult} type parameter.

\item \lstinline{OrderBy()} (and \lstinline{OrderByDescending()}) sets the ordering.
Because it doesn't make sense to sort the same query multiple times
and because no module supports sorting by multiple keys,
this method returns the type with \lstinline{Sortable} removed.

This method is available on the two \lstinline{Sortable} types
and uses the type parameter \lstinline{TOrderBy}.

\item \lstinline{Pages} is a property that returns a \lstinline{PagesSource}
that can then be used in a \texttt{prop} query.
See Section~\ref{PagesSource} for more information.

This property is available on the two \lstinline{Generator} types
and uses the type parameter \lstinline{TPage}.

\end{itemize}

For a state diagram of transitions between the \lstinline{WikiQuery} types and other related types,
see Figure~\ref{WikiQuery types}.

\begin{figure}[htbp]
\begin{center}
\begin{tikzpicture}[>=angle 90]
\path (0,9) node(WQS) {\lstinline{WQSortable}};
\path (5,9) node(WQ) {\lstinline{WQ}};

\path (5,6) node(WQR) {\lstinline{WQResult}};

\path (0,3) node(WQSG) {\lstinline{WQSortableGenerator}};
\path (5,3) node(WQG) {\lstinline{WQGenerator}};

\path (0,0) node(PS) {\lstinline{PagesSource}};
\path (5,0) node(WQPR) {\lstinline{WQPageResult}};

\draw[->] (WQS) edge [out=188,in=172,looseness=5,auto] node {\lstinline{Where}} (WQS);
\draw[->] (WQS) edge node[above] {\lstinline{OrderBy}} (WQ);
\draw[->] (WQS) edge node[below left] {\lstinline{Select}} (WQR);

\draw[->] (WQ) edge [out=-20,in=20,looseness=8,right] node {\lstinline{Where}} (WQ);
\draw[->] (WQ) edge node[right] {\lstinline{Select}} (WQR);

\draw[->] (WQSG) edge [out=185,in=175,looseness=5,auto] node {\lstinline{Where}} (WQSG);
\draw[->] (WQSG) edge node[auto] {\lstinline{OrderBy}} (WQG);
\draw[->] (WQSG) edge node[above left] {\lstinline{Select}} (WQR);

\draw[->] (WQG) edge [out=-8,in=8,looseness=5,right] node {\lstinline{Where}} (WQG);
\draw[->] (WQG) edge node[right] {\lstinline{Select}} (WQR);

\draw[->] (WQSG) edge node[auto] {\lstinline{Pages}} (PS);
\draw[->] (WQG) edge node[auto] {\lstinline{Pages}} (PS);

\draw[->] (PS) edge node[auto] {\lstinline{Select}} (WQPR);

\end{tikzpicture}
\end{center}

\caption{State diagram of \lstinline{WikiQuery}-related types \\ (\lstinline{WikiQuery} is shortened to \lstinline{WQ} to save space)}
\label{WikiQuery types}

\end{figure}

\subsection{\texorpdfstring{\lstinline{PagesSource}}{PagesSource}}
\label{PagesSource}

The \lstinline{PagesSource} type represents a collection of pages that can be used in \texttt{prop} queries,
to get information about those pages.
This information can be for example a list of categories for each page in the collection.

There are two kinds of \lstinline{PagesSource}s: generator-based and list-based.

\medskip

List-based sources use a static list of pages, given as a collection of page titles, page IDs or revision IDs.

Because the number of pages given this way in a single API request is fairly limited (usually to 50),
large lists have to be queried multiple times.
\lstinline{PagesSource} handles this transparently, so the user can input as many pages as he wants and doesn't have to worry about the limit.

One exception is if the limit is different than the default of 50 for the current user on the current wiki.
In that case, the user should change the limit by setting the static property \lstinline{ListPagesCollection.MaxLimit}.%
\footnote{In all other cases where limits are important in this library, they limit the output, not the input.
That is why simply setting \texttt{limit=max} works in those other cases, but doesn't work here.}

If the collection used to create a \lstinline{PagesSource} is lazy, it is iterated in a lazy manner.
For example, it could be the result of another LinqToWiki query, with additional processing by LINQ to objects,
that is not possible using LinqToWiki alone.
Or it could the result of a query from another wiki.
In such cases, the original query will only make as many requests as necessary for the follow-up query.

\medskip

Generator-based sources represent a dynamic list of pages that is the result of another API query,
like the list of all pages on a wiki from the \texttt{allpages} module.
This way, the list of pages doesn't have to be retrieved separately, only to be sent back.

Generator queries also have to handle paging, as described in Section~\ref{mw paging},
including the exception for the \texttt{revisions} module.

\medskip

Thanks to the fact that both kinds of page sources for \texttt{prop} queries are represented by the same
(abstract) type, the user of this library can use the same code to work with any source,
which could lead to repetitive code otherwise.

\medskip

To actually create a \texttt{prop} query for a page source, one uses the \lstinline{Select()} method.
Its parameter is a lambda, whose parameter is the type parameter \lstinline{TPage} of \lstinline{PagesSource}.
This type is the same for all queries on the same wiki, but could be different for differnt wikis.

Inside the lambda, properties and methods of the \lstinline{TPage} type can be accessed.
Each of them represents a \texttt{prop} module and all of the methods return one of the \lstinline{WikiQuery} types,
which can then be queried as usual, with one condition:
The \lstinline{WikiQuery} types can't ``leak'' outside of the query, so one has to use \lstinline{ToEnumerable()} or \lstinline{ToList()} inside the lambda.

If a \texttt{prop} module has a single result (not a collection), it is represented as a property
that directly returns this result, no querying is possible.

For an example of \lstinline{PagesSource} query, see Figure~\ref{PS query}.

\begin{figure}[htbp]

\begin{lstlisting}
pagesSource.Select(
    p =>
    new
    {
        p.Info,
        Categories =
            p.Categories()
            .Where(c => c.Show == Show.NotHidden)
            .Select(c => new { c.Title, c.SortKeyPrefix })
            .ToEnumerable()
            .Take(10)
	}
)
\end{lstlisting}

\caption{An example of  \lstinline{PagesSource} query that uses the \texttt{info} and \texttt{categories} modules}
\label{PS query}

\end{figure}

% PageSources and types related to prop queries
% ExpressionParser
% QueryParameters
% QueryProcessor, its helpers

\subsection{\texorpdfstring{\lstinline{Downloader}}{Downloader}}

% TODO: reword
Probably the most basic type in this project is \lstinline{Downloader}.
It takes care of forming the query string, executing the request and
returning the result as an \lstinline{XDocument}.
\lstinline{XDocument} is part of LINQ to XML, a part of .Net framework for manipulating XML documents.

\lstinline{Downloader} always uses POST and formats its requests as \path{application/x-www-form-urlencoded}.
This means that all modules work, including those that require POST.
On the other hand, uploads of files don't work, because they require \path{multipart/form-data}.

The decision to use \path{application/x-www-form-urlencoded} follows from the fact that
\path{multipart/form-data} is very inefficient when sending multiple parameters with short values,
which is common when making requests to the API.

% TODO: maybe some conclusion?

\section{The LinqToWiki.Codegen project}

\section{The linqtowiki-codegen application}

\section{Samples of queries}
\chapter{Future work}

% mention < and >?

While the LinqToWiki library is fully functioning and could be considered complete,
there is still room for improvement.
Some possible improvements were already mentioned
(proper capitalization in generated code;
distinguishing \lstinline{Where()} parameters from other parameters),
but there are also other improvements that could have larger impact on the usage of the library:

\begin{itemize}
\item \emph{F\# implementation}

The recent version 3.0 of the .NET-based functional language F\# has its own alternative to \ac{LINQ},
called query expressions.
Query expressions use their own set of types, so they are incompatible with C\# \ac{LINQ},
but they offer similar capabilities.
Unlike C\#, they also support creating custom operators,
which could provide better syntax for \lstinline{PagesSource} queries.

Also, F\# 3.0 supports type providers, which is a feature for automatic creating of types
just before they are used in code, that is, before compilation.
This would simplify the workflow of using LinqToWiki, because the code generation would be automatic.

\item \emph{Asynchrony}

The current version of LinqToWiki is completely synchronous,
which means there is always a thread blocked, while an application waits for a network response.
Better support for asynchrony is the main feature of the new C\#~5.0,
but LinqToWiki could be made asynchronous even without it.
This could be useful especially in \ac{GUI} applications, where the main thread should not be blocked for long periods of time.

A big part of LinqToWiki is working with modules that return lists,
but there is no single idiomatic way of representing asynchronous collections,
with or without C\# 5.0 improvements.
There are several possibilities, including a lazy collection of \lstinline{Task}s,
\acs{Rx} \lstinline{IObservable} or \acs{TPL} Dataflow \lstinline{ISourceBlock}.

Another way how asynchrony could be used in LinqToWiki is if the following page was being retrieved
even before the user finished processing the preceding page.
This could be especially useful for \lstinline{PagesSource} queries,
because the following primary and secondary page could be fetched in parallel.

\item \emph{Better compile-time checking}

One of the features of Roslyn that is not currently used by LinqToWiki is Visual Studio integration.
What that means is that one can use Roslyn to write various Visual Studio extensions,
including enhanced compile-time checking.

In the case of LinqToWiki, this could be used to verify that queries that compile are actually correct.
If not, a custom error would show in the list of errors.
Such checking would be also useful for other \acs{LINQ} providers, such as LINQ to SQL,
but each provider has its own rules about what is an error and what is a correct query.

% extensions?

\end{itemize}
\chapter*{Conclusion}
\addcontentsline{toc}{chapter}{Conclusion}

The goal of this work was to implement a C\# library to access the MediaWiki API
in a way that is readable, discoverable, strongly-typed and flexible.

These goals were successfully accomplished using a custom LINQ provider and code generation with Roslyn.

The actions available through the LINQ provider closely match those that are available from the API
by intentionally not supporting all LINQ operators, and using different set of properties for each operator.

TODO: continue here

%%% Seznam použité literatury
%%% Seznam použité literatury je zpracován podle platných standardů. Povinnou citační
%%% normou pro bakalářskou práci je ISO 690. Jména časopisů lze uvádět zkrácená, ale jen
%%% v kodifikované podobě. Všechny použité zdroje a prameny musí být řádně citovány.

\def\bibname{Bibliography}
\begin{thebibliography}{12}
\addcontentsline{toc}{chapter}{\bibname}

%\bibitem{lamport94}
%  {\sc Lamport,} Leslie.
%  \emph{\LaTeX: A Document Preparation System}.
%  2. vydání.
%  Massachusetts: Addison Wesley, 1994.
%  ISBN 0-201-52983-1.

\bibitem{mediawiki}
 \emph{MediaWiki}.
 MediaWiki.org.
 \url{http://www.mediawiki.org/wiki/MediaWiki}.

\bibitem{mediawiki-api}
 \emph{MediaWiki API}.
 MediaWiki.org.
 \url{http://www.mediawiki.org/wiki/API}.
 
\bibitem{cs-in-depth}
  {\sc Skeet,} Jon.
  \emph{C\# in Depth}.
  2nd edition.
  Stamford: Manning, 2011.
  Part 3, C\#~3: Revolutionizing how we code.
  ISBN 978-1-935182-47-4.
  
\bibitem{roslyn}
 \emph{Microsoft “Roslyn” CTP}.
 MSDN.
 \url{http://msdn.microsoft.com/en-US/roslyn}.
 
\bibitem{linq-to-sql-functions}
 \emph{Data Types and Functions}.
 LINQ to SQL Reference, MSDN Library.
 \url{http://msdn.microsoft.com/en-us/library/bb386970}.

\bibitem{warren}
 {\sc Warren}, Matt.
 \emph{LINQ: Building an IQueryable Provider -- Part III}.
 The Wayward WebLog.
 \url{http://blogs.msdn.com/b/mattwar/archive/2007/08/01/linq-building-an-iqueryable-provider-part-iii.aspx},
 2 August 2007.
 
\bibitem{reflection-emit}
 \emph{Emitting Dynamic Methods and Assemblies}.
 MSDN Library.
 \url{http://msdn.microsoft.com/en-us/library/8ffc3x75}.

\bibitem{codedom}
 \emph{Dynamic Source Code Generation and Compilation}.
 MSDN Library.
 \url{http://msdn.microsoft.com/en-us/library/650ax5cx}.
 
\bibitem{guidelines-for-names}
 \emph{Guidelines for Names}.
 Design Guidelines for Developing Class Libraries, MSDN Library.
 \url{http://msdn.microsoft.com/en-us/library/ms229002}.
 
\bibitem{wikitools}
 \emph{python-wikitools}.
 Google Code.
 \url{http://code.google.com/p/python-wikitools/}.
 
\bibitem{wikifunctions}
 \emph{WikiFunctions}.
 Wikipedia.
 \url{http://en.wikipedia.org/wiki/Wikipedia:WikiFunctions}.
 
\bibitem{linq-to-wikipedia}
 \emph{Linq to Wikipedia}.
 CodePlex.
 \url{http://linqtowikipedia.codeplex.com/}.
 
\end{thebibliography}

%%% Tabulky v bakalářské práci, existují-li.
\chapwithtoc{List of Tables}

%%% Použité zkratky v bakalářské práci, existují-li, včetně jejich vysvětlení.
\chapwithtoc{List of Abbreviations}

%%% Přílohy k bakalářské práci, existují-li (různé dodatky jako výpisy programů,
%%% diagramy apod.). Každá příloha musí být alespoň jednou odkazována z vlastního
%%% textu práce. Přílohy se číslují.
\chapwithtoc{Attachments}

\openright
\end{document}